\documentclass{article}
\usepackage[cm]{fullpage}
\usepackage{graphicx}
\usepackage{amsmath}
\usepackage{amssymb}
\usepackage{enumerate}
\usepackage{float}
\restylefloat{table}
\usepackage{listings}
\usepackage{color}
\usepackage{fancybox}
\usepackage{caption}
\usepackage{subcaption}
\usepackage[USenglish]{isodate}
\isodate
\usepackage{fancyhdr}
\pagestyle{fancy}
\headheight 10pt
\headsep 10pt
\rhead{\today}
\chead{EE 141 Project}
\lhead{Nathan Aclander, Troy Sankey, and Sakib Shaikh}
\cfoot{\thepage}

\title{Hard Disk Drive (HDD) Read Header Controller Design}
\date{\today}
\author{Nathan Aclander (903933664), Troy Sankey (403942345), 
and Sakib Shaikh (703940302)}

\newcommand{\matlab}[1]{%
\lstinputlisting[language=Matlab,
                 breaklines=true,
                 morekeywords={matlab2tikz},
                 numbers=left,
                 numberstyle={\tiny \color{black}},
                 keywordstyle=\color{blue},
                 showstringspaces=false,
                 numbersep=9pt,
                 xleftmargin=3em
                ]{#1}%
}

\begin{document}

\maketitle
\newpage

\section*{Summary of Objectives}

\section*{Project Tasks}
\subsection*{Task 1: Deriving the Transfer Function}

In this task we are to derive the transfer function $G_2(s) =
\frac{Y(s)}{U(s)}$ of the head reader position model. This relates the
input torque $u(t)$ to the head position $y(t)$. To do this we will
use the Laplace Transform as learned in class to get the desired
transfer function.

We know our initial conditions for $\dot{y}$ and $\ddot{y}$ are $0$
because the HDD head is not in motion.

\begin{align*}
  \dot{y} &= 0 \\
  \ddot{y} &= 0 \\
  \mathcal{L}\left\{ J\ddot{y} + b \dot{y}\right\} &= \mathcal{L}\left\{u(t)\right\} \\
  Js^2 Y(s) + bsY(s) &= U(s) \\
  G_2 &= \frac{Y(s)}{U(s)} = \frac{\frac{1}{b}}{s(\frac{J}{b}s + 1)} \\
\end{align*}

Calculating the Laplace Transform of the torque to position
relationship was relatively easy as shown above. We used differential
equation solving techniques learned in class, and are now ready to
apply this transfer function in the remaining tasks of this project.

\subsection*{Task 2: Calculating the Open-Loop Transfer Function}

In this task we are to calculate the open-loop transfer function of
the cascaded HDD head reader assembly, and additionally obtain the
unit step response plot. The first function of our cascaded system was
given to us in the project specification. It is the transfer function
of the motor coil that relates the input voltage to the output torque,
it is shown below:

$$G_1(s) = \frac{U(s)}{V(s)} = \frac{K_m}{Ls + R}$$

To calculate the open-loop transfer function we relied on the fact
that the cascaded transfer function is $G_1$ and $G_2$ multiplied
together, as learned in class. The cascaded open-loop transfer
function is shown below:

\begin{align*}
  G_1G_2 &= \frac{K_m}{Ls+R} \cdot \frac{\frac{1}{b}}{s(\frac{J}{b}s + 1)} \\
         &= \frac{\frac{K_m}{bR}}{s\left(\frac{J}{b}s + 1\right)  
		 \left(\frac{L}{R}s + 1\right)} \\
         &= \frac{1}{\frac{JL}{K_m}s^3 + s^2\left( \frac{RJ}{K_m} + 
		 \frac{bL}{K_m} \right) + \frac{bR}{K_m}s}
\end{align*}

The second part of this task asked us to take the unit step response
of the open-loop transfer function we just derived. The project
specification mentioned to use the values from the table below:

\begin{table}[H]
  \begin{center}
    \begin{tabular}{ | l | l | l | p{5cm} |}
    \hline
    \textbf{Parameter} & \textbf{Symbol} & \textbf{Typical Value} \\ \hline
    Inertia of arm and head & $J$ & 1 N $\cdot$ m s$^2$/rad \\ \hline 
    Friction & $b$ & 20 kg/m/s \\ \hline
    Field Resistance & $R$ & 1 $\Omega$ \\ \hline
    Field Inductance & $L$ & 0.001 H \\ \hline
    Motor Constant & $K_m$ & 1 N $\cdot$ m/A \\ \hline
   \end{tabular}
 \end{center}
 \caption{Model Parameters}
\end{table}

After pluging in our values from the table into our transfer function
equation we get the following:

$$\frac{1}{0.001s^3 + 1.02 s^2 + 20s}$$

And our Octave code and plot of the impulse response is shown below:

\matlab{fig1.m}

\begin{figure}[H]
  \caption{Open-loop transfer function of the cascaded HDD head reader
    assembly}
  \centering
  \includegraphics[width=0.5\textwidth]{fig1.eps}
\end{figure}

In conclusion we were able to derive and expression for the open-loop
transfer function by cascading the head reader position model and the
motor coil transfer function together. We then observed, using Octave,
the step response on our new transfer function using the provided
model parameters.

\subsection*{Task 3: Proportional Compensator}

This task required us to find the step response of the closed-loop system to
the unit step refrence input, while assuming zero disturbance. To complete the
closed loop we were given the sensor's transfer function $H(s) = 1$ and the
candidate controller $F(s) = K_a$. This task relies on our understanding on
block diagrams as explained in class, as well as application of compensator
design given transient response performance specifications.

\subsubsection*{Task 3A: Responses: Unit Step Refrence Input}

%TODO Maybe show block diagram?

Our closed-loop is represented as: 

$$HDD(s) = \frac{F(s)G_1(s)G_2(s)}{1 + F(s)G_1(s)G_2(s)}$$

\noindent
where $G_1(s)G_2(s)$ is the transfer function from part 2. Using the two
specified values for $K_a$ we get the following transfer function when using
values from Table 1.

\begin{table}[H]
\begin{center}
  \begin{tabular}{ | l | l | l | p{5cm} |}
  \hline
  \textbf{$K_a$} & \textbf{$HDD(s)$}  \\ \hline
  50 & $\frac{0.05s^3 + 41s^2 + 1000s}{0.00204s^5 1.08s^4 + 40.85 s^3 
  + 451 s^2  + 1000s}$\\ \hline 
  400 & $\frac{0.4s^3 + 408s^2 + 8000s}{0.00204s^5 + 1.08s^4 + 41.2 s^3
  + 808 s^2 + 8000 s}$  \\ \hline
 \end{tabular}
\end{center}
\caption{Closed-Loop System: Unit Step Refrence Input}
\end{table}

Using Octave, we observed the response to the unit step refrence input. The
Octave code and plots are shown below. 

\matlab{fig2.m}

\begin{figure}[H]
  \caption{Step response for the closed loop system: $K_a = 50$}
  \centering
  \includegraphics[width=0.5\textwidth]{fig2.eps}
\end{figure}

\matlab{fig3.m}

\begin{figure}[H]
  \caption{Step response for the closed loop system: $K_a = 400$}
  \centering
  \includegraphics[width=0.5\textwidth]{fig3.eps}
\end{figure}

\subsubsection*{Task 3B: Responses: Unit Step Disturbance}

%TODO Maybe show block diagram?

Our closed-loop is represented as: 

$$ HDD(s) = \frac{G_2(s)}{1 + G_1(s)H(s)F(s)G_2(s)} $$ 

Again, using the specifications for part 2 we get:

\begin{table}[H]
\begin{center}
  \begin{tabular}{ | l | l | l | p{5cm} |}
  \hline
  \textbf{$K_a$} & \textbf{$HDD(s)$}  \\ \hline
  50 & $\frac{0.00255s^2 + 0.05s}{0.0026s^4 + 0.101s^3 + 1.125s^2 + 2.5s}$\\ \hline 
  400 & $\frac{0.00255s^2 + 0.05s}{0.0026s^4 + 0.101s^3 + 2s^2 + 20s}$\\ \hline 
 \end{tabular}
\end{center}
\caption{Closed-Loop System: Unit Step Disturbance}
\end{table}

Using Octave, we observed the response to the unit step refrence input. The
Octave code and plots are shown below. 

\matlab{fig4.m}

\begin{figure}[H]
  \caption{Step response for the closed loop system: $K_a = 50$}
  \centering
  \includegraphics[width=0.5\textwidth]{fig4.eps}
\end{figure}

\matlab{fig5.m}

\begin{figure}[H]
  \caption{Step response for the closed loop system: $K_a = 400$}
  \centering
  \includegraphics[width=0.5\textwidth]{fig5.eps}
\end{figure}

\subsubsection*{Task 3C: Satisfying the Performance Specifications}

For this subtask, a value for $K_a$ was needed to meet performance 
specifications of the system. The specifications are shown inthe table below.

\begin{table}[H]
\begin{center}
  \begin{tabular}{ | l | l | l | p{5cm} |}
  \hline
  \textbf{Performance Measure} & \textbf{Specification}\\ \hline 
  Percent overshoot & Less than $5\%$ \\ \hline 
  Settling time ($2\%$ deviation) & Less than 250 ms \\ \hline 
  Maximum Value of response to a unit step disturbance & Less than 
  $5\cdot10^{-3}$ \\ \hline 
 \end{tabular}
\end{center}
\caption{Transient Response Performance Specifications}
\end{table}

After observing the response of the system for varying values of $K_a$ we
concluded that it was not possible to meet all aspects of the specifications
from the proportional compensator. 

The shortest settling time we could get was approximatley 0.37 seconds, for
$K_a = 1450$. We noticed a parabolic relationship between $K_a$ and the
settling time. Values smaller, and larger than $K_a = 1450$ gave longer
settling times. Settling time seemed to be the smallest at approximatley
$K_a = 1450$.

\matlab{fig6.m}

\begin{figure}[H]
  \caption{Shortest Settling Time Acheivable From Compensator: 
  $K_a \approx 1450$, Settling Time $\approx 370$ ms}
  \centering
  \includegraphics[width=0.5\textwidth]{fig6.eps}
\end{figure}

To meet the appropriate overshoot, we set $K_a$ to 200, which
indeed did place the system within the correct specification margins for
overshoot, but placed it outside the bound of the maximum value of the response
 to a unit step disturbance. Increasing $K_a$ higher brought the system
closer to meeting the specification for the maximum bound, but further away from
meeting the overshoot specification. $K_a = 220$ for example, gave a maximum
value just under the required $5\times 10^{-3}$.

The following graphs show how changing $K_a$ from 200 to 220 pushes
the response of the unit step disturbance over the boundary of its
accepted maximum response.

\begin{figure}\centering
  \begin{minipage}{0.5\linewidth}
    \matlab{fig7.m}%
  \end{minipage}%
  \begin{minipage}{0.5\linewidth}
    \matlab{fig8.m}
  \end{minipage}%
\end{figure}

\begin{figure}[H]\centering
  \caption{step responses for different $K_a$}
  \begin{minipage}{9cm}
    \includegraphics[width=1.0\textwidth]{fig7.eps}
  \end{minipage}%
  \begin{minipage}{9cm}
    \includegraphics[width=1.0\textwidth]{fig8.eps}
  \end{minipage}
\end{figure}

\begin{figure}[H]
  \caption{Root Locus Plot for HDD(S): We see that for any value of $K$ 
  (highlighted line) we cannot meet all of the requirements}
  \centering
  \includegraphics[width=0.5\textwidth]{fig13.eps}
\end{figure}

In conclusion, this task required building a complete mathematical model for
the HDD Read Head system, and then using this model to meet required
specifications. This task also showed, however, that adding a compensator will
not neccessarilly change the system to meet all specifications. In part C of
this task, we saw that when adding a compensator we were able to meet certain
specification requirements, but never all of them. This task made an important
observation in showing not all compensators are the same, as tasks later on
will show.

\subsection*{Task 4: Positional and Velocity Sensor}

In this task an alternative output sensor was considered, that measures the
position as well as the velocity of the head reader. Its represented by $H(s)
= 1 + K_1s$. This task, like Task 3C, relied in our understanding from class of
manipulating transfer functions with compensators. Unlike Task 3C, we now have
control over both poles and zeroes. Our system transfer function is now the 
same as in task 3A, but with a different value for $H(s)$:

$$HDD(s) = \frac{F(s)G_1(s)G_2(s)}{1 + F(s)G_1(s)G_2(s)H(s)}$$

The system now has two different K values that can be changed in order to meet
the specifications from Table 4. After trying different K values we found that
$K_a = 370$ and $K_1 = 0.04$ enabled the system to satisfy all requirements.
A value of $K_1 = 0.04$ gave our system a zero at $-25$.

Using these values, the system shows the following values in its response:

\begin{table}[H]
\begin{center}
  \begin{tabular}{ | l | l | l | p{5cm} |}
  \hline
  \textbf{Performance Measure} & \textbf{Specification}\\ \hline
  Percent overshoot & 0.134\% \\ \hline
  Settling time ($2\%$ deviation) & 243ms \\ \hline
  Maximum Value of response to a unit step disturbance & $3\cdot10^{-3}$\\ \hline 
 \end{tabular}
\end{center}
\caption{Response Performance from Positional Velocity Sensor}
\end{table}

All above values were within the specifications of the system, and all were
achieved with the same K values, inlike the system in Task 3.

%TODO show graphs that help explain?

In conclusion, it was also found that for this system, having a zero less than
-20 enabled the creation of a system that was able to meet the specifications.
This was because a zero at that particular location caused the poles of the
system to break out towards the left, and into the desired regioun (as set by
the specifications).

This task showed how having control of both zeros and poles of the system gave
the engineer of the system (us) more control. Without the ability to add a
zero to the system and changing its position, the system would not have been
able to meet the specifications, as was the case in Task 3C.

\subsection*{Task 5: The PID Compensator}

In this task we used the PID Compensator architectire. The sensor reverted back
to $H(s) = 1$, while the controller now assumes a transfer function of the form

$$ F(s) = K_1 + \frac{K_2}{s}  + K_3s $$

\noindent
where $K_2 = 0$ because the plant (arm) inherently has an integrating
term. The project description furthermore states that $\frac{K_1}{K_2}
= 1$. As in the previous Tasks, our system transfer function is still
of the form:

$$HDD(s) = \frac{F(s)G_1(s)G_2(s)}{1 + F(s)G_1(s)G_2(s)H(s)}$$

Using the PID Compensator the system was not able to acheive a settling time
of less than 250 ms, as described in the spec. However, other design
specifications were able to be met. The following table the \% overshoot,
response time, and response to disturbance for varying levels of $K$. Note that
in this compensator both $K_1$ and $K_3$ were the same, as their ratio was
required to be 1.

\begin{table}[H]
  \begin{center}
    \begin{tabular}{ | r | r | r | p{5.5cm} |}
    \hline
    $K_1$ and $K_3$ & \textbf{\% Overshoot} &
        \textbf{Settling Time (sec)} & \textbf{Response to Step Disturbance} \\ \hline
    10000 & 0 & 3.92 & $1e^{-4}$ \\ \hline
    500 & 0 & 4.06 & $2e^{-3}$ \\ \hline
    100 & 0 & 4.50 & 0.01 \\ \hline
    1 & 0 & 82.00 & 1 \\ \hline
    \end{tabular}
  \end{center}
  \caption{Response Performance from PID Compensator}
\end{table}

%TODO show graphs that help explain?

\subsection*{Task 6: Refined Arm Model and Frequency Response Characterization}

This is the transfer function for the spring load system:

$$G_3 = \frac{1}{1 + \frac{2\zeta s}{\omega_n} + \frac{s^2}{\omega_n^2}}
      = \frac{1}{2.814\times 10^{-9} ~ s^2 + 3.183\times 10^{-5} ~ s + 1}$$

\matlab{fig9.m}

\begin{figure}[H]
  \caption{Step response of spring load system}
  \centering
  \includegraphics[width=0.5\textwidth]{fig9.eps}
\end{figure}

The closed loop system transfer function changes to the following
form:

$$HDD(s) = \frac{F(s)G_1(s)G_2(s)G_3(s)}{1 + F(s)G_1(s)G_2(s)G_3(s)H(s)}$$

\noindent
where $H(s) = 1$.  The parameter $K_1 = K_3$ is to be varied from 100
to 1000, while the Bode plot of the closed loop system is observed
below.

\matlab{fig10.m}

\begin{figure}[H]\centering
  \caption{bode plots for varying $K_1 = K_3$}
  \begin{minipage}{9cm}
    \includegraphics[width=1.0\textwidth]{fig10.eps}
  \end{minipage}%
  \begin{minipage}{9cm}
    \includegraphics[width=1.0\textwidth]{fig11.eps}
  \end{minipage}\\
  \begin{minipage}{9cm}
    \includegraphics[width=1.0\textwidth]{fig12.eps}
  \end{minipage}
\end{figure}


\end{document}
